% !TeX spellcheck = ru_RU
% !TEX root = vkr.tex

\section*{Введение}
\thispagestyle{withCompileDate}

\verb|TATLIN.BACKUP|\footnote{\href{https://yadro.com/ru/tatlin/backup}{TATLIN.BACKUP}
--- Уникальная система хранения данных резервных копий} --- проект группы кампаний \verb|YADRO|\footnote{\href{https://yadro.com/}{YADRO} --- официальный сайт компании}, посвященный созданию системы бекапов с использованием алгоритмов дедупликации написанный на языке \verb|Rust|. \verb|Rust| предоставляет \verb|async| / \verb|await| интерфейс для обработки асинхронных событий, при этом язык не фиксирует реализацию, позволяя пользователю выбирать асинхронный рантайм.

\verb|tokio|\footnote{\href{https://tokio.rs/}{tokio} --- официальный сайт проекта} --- проект, предоставляющий две реализации асинхронного рантайма: так называемые \verb|current_thread| и \verb|mutli_thread| рантаймы, частично реализует стандартную библиотеку языка с асинхронным интерфейсом, примитивы синхранизации, коллекции, таймеры, профилировщики --- одним словом целую экосистему для написания асинхронных программ.

Многопоточный рантайм из \verb|tokio| используется в \verb|TATLIN.BACKUP|, где был замечен недостаток текущей реализации: общая очередь асинхронный событий, разделяемая всеми потоками \verb|tokio|, защищена мьютексом и становится бутылочным горлышком по предположению инженеров из \verb|YADRO|.

Именно исследованию возможности улучшения производительсности многопоточного рантайма \verb|tokio| в типичных для \verb|TATLIN.BACKUP| сценариях с помощью изменения алгоритмов общей очереди событий посвящена данная работа.
