% !TeX spellcheck = ru_RU
% !TEX root = vkr.tex

\section*{Введение}
\thispagestyle{withCompileDate}

% \begin{enumerate}
%     \item Асинхронное программирование (что это, зачем нужно, как часто встречается???)
%     \item Tatlin backup and their problems
%     \item Хочется улучшить.
%     \begin{itemize} move this to task section
%         \item Пишем бенчмарки
%         \item Собираем результаты
%         \item Изменить алгоритм так, чтобы улучшились интересующие нас бенчмарки
%         \item Произвести сравнений измененного алгоритма в real word сценариях
%     \end{itemize}
% \end{enumerate}

При создании большой системы, будь то распределенное приложение или система на кристале, разработчики сталкиваются с пространственными ограничениям, которые не позволяют разным компонентам этой системы взаимодействовать синхронно~TODO(). Таким обрзом разработка сложных современных, все чаше распределенных, систем подразумевает асихронное взаимодейтсвие между компанентами этих систем, что само по себе увеличивает их сложность~TODO().

Дабы облегчить написание программ обрбатывающих асинхронные события современные языки программирования содержат поддержку написания асихронного кода на уровне языка: \verb|async|, \verb|await| в F\#, C\#, Rust, корутинны в Kotlin, горутуны в Go, корутины в C++23.

Не стал исключением вsше упомянутый язык Rust, в коем асихронные конструкции не только встроены на уровне синтаксическом, но позвоялет оперделять и использовать различные реализации асихронных рантаймов. То есть дизайн языка позволяет пользователю менять саму семантику обработки асихронных событий, при этом без изменения исходного кода программы.

Напрмиер, tokio --- один из самых популярных и продвинутых проектов~TODO() в этом направлении предоставляет две реализации асинхронного рантайма: так называемые \verb|current_thread| и \verb|mutli_thread| рантаймы, и частично реализует стандартную библиотеку языка с асинхронным интерфейсом. А так же предоставляет собственные примитивы синхранизации, коллекции, таймеры, профилировщики --- одним словом целую экосистему для написания асинхронных программ.

Именно реалиизация многопоточного рантайма используется в проекте Tatlin Backup, именно там были обнаружены недостатки этой реализации.
