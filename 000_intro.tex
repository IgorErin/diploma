% !TeX spellcheck = ru_RU
% !TEX root = vkr.tex

\section*{Введение}
\thispagestyle{withCompileDate}

% \begin{enumerate}
%     \item Асинхронное программирование (что это, зачем нужно, как часто встречается???)
%     \item Асинхронное программирование в rust.
%     \item Хочется улучшить. (Проблма: Нет публичных доказательств того, что это действительно нужно -> посроить завести issue???)
%     \begin{itemize} move this to task section
%         \item Пишем бенчмарки
%         \item Собираем результаты
%         \item Изменить алгоритм так, чтобы улучшились интересующие нас бенчмарки
%         \item Произвести сравнений измененного алгоритма в real word сценариях
%     \end{itemize}
% \end{enumerate}

Асинхронные задачи в программировании встречаются повсеместно: пользовтельские интерфейсы, облачные вычисления, взаиммодействие с файловой системой~\cite{ProAsyncDotNet}.
В следствие чего, важность преобретает производительная поддержка асинхронных возможностей на уровне языка, рантаймой и библиотек~\cite{AsyncRustProg}.

Язык \verb|Rust| предоставляет возможность писать асихнронный код по средству async/await синтаксиса и позволяет выбирать асинхронный рантайм как внешнюю бибилиотеку. Одним из самых популярных рантаймов общего назначения является tokio TODO(proof). Tokio предоставляет рантайм, набор асихронных бибилиотек, профиляровщики и прочее, TODO(rewrite) что делает этот проект незаменимым.

Плохая приспособленность шедулера tokio под задачи команды tatlin.backup не позволяет пользоваться всей экосистеймой проекта.