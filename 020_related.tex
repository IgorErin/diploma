% !TeX spellcheck = ru_RU
% !TEX root = vkr.tex

\section{Обзор}

Данный раздел содержит обзор основных сущностей, взаимодейсвующих с глобальной очередью, выбор метрик рантайма для дальнейшего анализа, обоснование выбора инструментов бенчмаркинга.

\subsection{Многопоточный рантайм tokio}

Рантайм инициализируется фиксированным числом \verb|воркеров| --- сущностей отвественных за исполнение асинхронных задач. С каждым воркером ассоциирован отдельный системный поток и так называемоя ядро воркера --- структура содержащая состояние воркера. Например, там находится локальная очередь воркера.

\verb|Локальная очередь| воркера выступает в качестве кэша его задач, имеет фиксированный размер и предполагает добавление задач в нее исключительно из потока воркера, что владеет этой очередью. Однако, изъятие из нее может быть осуществленно потоками других воркеров при нехватке оным собственных задач. В случае переполнения локальной очереди часть задач перемещается воркером в глобальную очередь, также известную как \verb|inject queue|.

\verb|Inject queue| или глобальная очередь представляет собой коллекцию задач ожидающих исполнения. Наполняется она при спавнинге задачи вне контекста воркера или при переполеннии локальной очереди самих воркеров. Реализована с помощью интрузивного связного списка, защищенного мьютексом.

Когда пользователь хочет запустить задачу, он вызывает метод \verb|tokio::spawn|. Сделать это необходимо в контексте рантайма: внутри метода \verb|block_on| или в промежутке жизнии объекта \verb|EnterGuard|. Контекст реализован с помощью локальной для потока переменной, которая содержит указатели на внутренние структуры рантайма. Позже, задача помещается либо в локальную очередь воркера, если задача спавнилась в потоке воркера, либо в глобальную очередь рантайма.

% \newpage

% \subsection{Цикл работы воркера}

% Воркер в цикле пытается завладеть задачей для исполнения. Если ему это не удается, он пробует украсть задачу у других воркеров. Если и это ему не удается, то tokio паркует поток. На каждой итерации отсчитыается тик --- некая мера времени необходимая для распределения работы, как будет показано далее.

% Далее приведен упрощенный алгоритм работы воркера. После чего будут рассмотренны алгоритм взаимодейсвия с очередьми.

% \begin{minted}{rust}
fn run() {
    while !is_shutdown() {
        core.tick();
        if let Some(task) = core.next_task() {
            self.run_task(task, core);
            continue;
        }
        if let Some(task) = core.steal_work() {
            self.run_task(task, core);
            continue;
        }
        self.park(core)
    }
}
% \end{minted}

% \subsubsection{Выбор следущей задачи}

% Раз в 61 тик он пытается взять задачу из глобальной очереди. Если ему это не удается, он пытается взять задачу из локальной очереди и, каковым не был бы результат этой попытки, вовращает ее.

% Иначе, он пробует взять задачу из локальной очереди. В случае неудачи он пробует взять из глобальной очереди как минимум одну задачу, при этом старается взять сразу несколько, но не все, чтобы оставить другим воркерам. После чего добавляет взятые задачи в собственную локальную очередь, кроме одной, что сразу окажется принятой на исполнение.

% Попытка завладеть задачей может быть представлена следующим, снова упрощенным, кодом:

% \begin{minted}{rust}
fn next_task(&mut self) -> Option<Notified> {
    if self.tick % self.global_queue_interval == 0 {
        return next_remote_task().or_else(|| self.next_local_task())
    }
    let maybe_task = self.next_local_task();
    if maybe_task.is_some() {
        return maybe_task;
    }
    if inject().is_empty() {
        return None;
    }
    let n = min(
        inject().len() / WORKER_COUNT + 1,
        self.run_queue.remaining_slots(),
        self.run_queue.max_capacity() / 2,
    );
    let mut tasks = inject().pop_n(max(1, n));
    let ret = tasks.next();
    self.run_queue.push_back(tasks);
    return ret
}
% \end{minted}

\subsection{Метрики}

\verb|tokio-metrics| --- проект, предоставющий интерфейс для получения метрик рантайма. Далее будет представлен полный перечень метрик доступных из tokio-metrics. Метрики будут разбиты на группы по аналогии для уменьшения повторений.

Группа предполагает перечисление в следующем порядке: значение представленное в tokio-metrics как общее значение для всех воркеров, минимум и максимум, оданко для краткости в тексте будут обзоначены только именования общих занчений.

\begin{itemize}
    \item \verb|total_park_count| --- количество парковок потоков воркеров
    \item \verb|mean_poll_duration| --- это значение представляет собой экспоненциально взвешенную скользящую среднюю продолжительности опросов задач
    \item \verb|total_noop_count| --- сколько раз поток воркера был распаркован, но не совершил никакой работы перед парковкой
    \item \verb|total_steal_count| --- количество задач, которые воркеры похитили, перместив их в свою локальную очередь
    \item \verb|total_steal_operations| --- количесво раз воркеры успешно похител задачи
    \item \verb|total_local_schedule_count| --- количество задач отправленных на исполенние из контектса воркера, что должна попасть в одну из локальных очередей
    \item \verb|total_overflow_count| --- сколько раз воркеры переполнили свои локальные очереди
    \item \verb|total_polls_count| --- количество опросов задач среди
    \item \verb|total_busy_duration| --- количество времени исполения задач
    \item \verb|total_local_queue_depth| --- количество задач помещенных в локальные очереди
\end{itemize}

И остальные:

\begin{itemize}
    \item \verb|workers_count| --- количество воркеров исполняющих задачи. Значение специфицируется при инстанциации рантайма

    \item \verb|poll_time_histogram| --- гистограмма опросов задач, сгруппированная по времени исполнения опросов

    \item \verb|num_remote_schedules| --- количество задач, отправленных на исполнение из вне. То есть количество задачи заспавненных вне контекста воркера, задач, попавших в глобальную очередь

    \item \verb|global_queue_depth| --- количество задач, помещенных в глобальную очередь
\end{itemize}

Для решения поставленных задач были выделены метрики, отражающие взаимодействие воркеров с очередями. То есть, метрики, демонстрирущие глубину глобальной очереди (\verb|global_queue_depth|), локальных очередей (\verb|total_local_queue_depth|), количество переполнений (\verb|total_overflow_count|), количество похищенных задач (\verb|total_steal_count|), количество удаленных спавнов (\verb|num_remote_schedules|).

\subsection{Бенчмаркинг}

Для измерений времени исполнения и пропускной способности была использована библиотека \verb|criterion|\footnote{\href{https://github.com/bheisler/criterion.rs}{Репозиторий} проекта criterion}. Так как она популярна, имеет обширную документацию, а так же уже использована в \verb|tokio|.

\subsection{Изменения шедулера языка Go}

Как было отмечено ранее, алгоритмы шедулинга в tokio были созданы с оглядкой на реализацию рантайма языка Go. В свою очередь в шедулинг корутин в Kotlin был сделать по мотивам tokio. С тех пор, никаких изменений рантайм Go не претерпел, точно так же, как рантайм языка Kotlin.
