% !TeX spellcheck = ru_RU
% !TEX root = vkr.tex

\section{Заключение}

В ходе работы были выполнены следующие задачи:

\begin{itemize}
    \item Проведен обзор метрик рантайма
    \item Создан проект содержащий бенчмарки, позволяющий автоматически собирать метрики и визуализировать результаты
    \item Произведен эксперимент основанный на ключевом сценарии использования \verb|tokio| в \verb|TATLIN.BACKUP|
\end{itemize}

Был произведен ряд исследований направленный на обнаружение предполагаемой проблемы, в ходе которых ее существованием не было опровергнуто, напротив, было выявлено падением производительности, причина коего может быть согласована с упомянутой гипотезой.

Нельзя отрицать: производство листовых задач в блокирующих потоках требует эксклюзивного обращения к очереди для каждой задачи. Потоки воркеров снабжены локальными очередями, что буферизируют добавление задач в глобальную очередь и позволяют делиться с другими воркерами минуя захват глобального мьютекса при их продукции, чем не могу похвастаться блокирующие потоки. Однако, тяжело отрицать принадлежность процесса продукции большого количества листовых задач к блокирующим --- множество аллокаций, взаимодействие с мьютексом глобальной очереди и lock free алгоритмом добавления локальной имеют недетерминированное время исполнения. Таким образом, блокирующие потоки, лишенные буферизации при продукции задач, несут угрозу для производительности всей системы при исполнении натурального для них сценария.

Следующий шаг данной работы будет направлен на исследования возможности буферизации задач при их производстве в блокирующих потоках: какие накладные расходы несет добавление задач в глобальную очередь по одной относительно буферизации с возможности похищения и без.
