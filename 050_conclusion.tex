% !TeX spellcheck = ru_RU
% !TEX root = vkr.tex

\section{Заключение}

В ходе работы были выполнены следующие задачи:

\begin{itemize}
    \item Получено первое описание реализации tokio с точки зрения асинхронного интерфейса языка \verb|Rust|. В результате обзора выявлены две возможные уязвимости с точки зрения производительности: взаимодействие исполнителей и реализация глобальной очереди. Первая уязвимость была устранена в результате текущей работы путем шардирования планировщика многопоточного рантайма. Изменение многопоточного рантайма с точки зрения алгоритмов глобальной очереди рассмотрены в ВКР Матвея Калашникова: ``Оптимизация системы управления очередью задач в библиотеке Tokio для асинхронного рантайма''.
    \item Создан проект, содержащий бенчмарки, позволяющий автоматически собирать метрики и визуализировать результаты\footnote{\href{https://github.com/IgorErin/tokiobench}{Репозиторий} проект tokiobench (Дата обращения: 25.5.2025)}.
    \item Произведен эксперимент, основанный на ключевом сценарии использования \verb|tokio| в \verb|TATLIN.BACKUP|. Обнаружена возможность увеличить пропускную способность \verb|TATLIN.BACKUP|, что может свидетельствовать о существовании ограничения, накладываемого глобальной очередью. Однако, такой подход остается небезопасным.
    \item Предложена и реализована безопасная модификация рантайма, позволяющая использовать несколько глобальных очередей в одном инстансе рантайма\footnote{\href{https://github.com/IgorErin/tokio/pull/3}{Реализация} шардирования многопоточного рантайма. Имя пользователя: IgorErin (Дата обращения: 25.5.2025)}.
    \item Проведены эксперименты на целевом сценарии \verb|TATLIN.BACKUP|, показавшие увеличение производительности предложенного решения.
\end{itemize}

Языковые асинхронные интерфейсы трудны в создании вследствие необходимости наличия высокопроизводительной реализации для различных сред: будь то приложения для встроенных систем~\cite{CPPCoroutinesOnMicrocontrollers, AsyncIOT} или пользовательские программные продукты, способные эффективно исполняться на различных архитектурах и операционных системах~\cite{CPPCoroutinesDesignAndImpl}.

Известно множество исследований, направленных на изучение создания и оптимизации пуллов для исполнения вычислительных задач~\cite{ThreadPoolSize, PerfDeviationsInThreadPools, SyncInThreadPools, ProduserConsumerThreadPool}. Однако автору не удалось найти академические работы, посвященные реализации языковым асинхронных рантаймов. Вероятно, это связано с относительно недавним их появлением в языках с повышенным контролем над ресурсами~\cite{CPPCoroutinesDesignAndImpl}.

Стоит отметить, что асинхронные движки таких популярных языков, как Go, Kotlin, tokio в экосистеме языка Rust представляют собой вариации дизайна, предложенного Дмитрием Вьюковым для языка Go~\cite{GoScheduler, GoSchedulerImpropvements}. Данная работа показывает, что незначительные изменения в этом популярном подходе могут увеличить пропускную способность отдельных бенчмарков на порядок, что может свидетельствовать о необходимости дальнейших исследований использования современных асинхронных интерфейсов~\cite{ModernStorageAPI} и подходов к многопоточности~\cite{ThreadPerCore}.
