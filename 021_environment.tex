\section{Условия экспериментов}

Для измерения производительности была использована библиотека \verb|criterion|\footnote{\href{https://github.com/bheisler/criterion.rs}{Репозиторий} проекта criterion (Дата обращения: 4.1.2025)}. Так как она популярна, имеет обширную документацию и используется в \verb|tokio|.

Здесь и далее эксперименты производились при следующих условиях:

\begin{itemize}
    \item Исследования проводились на системе YADRO VEGMAN Rx20 G2\footnote{\href{https://yadro.com/ru/vegman/rx20g2/specs}{Описание} системы YADRO VEGMAN Rx20 G2}.
    \item Бенчмарк был запущен с значением \verb|nice| равным минус двадцати.
    \item Исполнение было рекомендовано на одной NUMA единице с помощью \verb|taskset|.
\end{itemize}

Машина для измерения производительности была предоставлена командой \verb|TATLIN.BACKUP| и использовалась удаленно, в связи с чем на ней было невозможно отключение сети.

Библиотека criterion сконфигурирована следующим образом:

\begin{itemize}
    \item Пять секунд прогрева.
    \item Сорок семплов для каждого измерения.
    \item Линеаризация в качестве способа семплирования.
\end{itemize}
