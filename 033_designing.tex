% !TeX spellcheck = ru_RU
% !TEX root = vkr.tex

\section{Проектирование решения}

После полученных результатов команда \verb|TATLIN.BACKUP| провела эксперименты, где так же стало очевидно преимущество использования нескольких рантаймов с точки зрения пропускной способности.

Однако, неясным осталась надежность такой системы: разработчики tokio не дают никаких гарантий при исполоьзовании нескольких рантаймов, а наоборот утверждают, что при некотором обращении с такой конструкцией могут быть наблюдаемы падения производительности или достижимо неопределенное поведение.

\verb|TATLIN.BACKUP| позиционирует себя как высоконадежное производительное решение, а потому использовать в нем конструкцию без гарантий не возможно.

\subsection{Шардирование глобальной очереди}

Если дублировать весь рантайм нельзя, вероятно, уменьшить ограничение накладовое глобальной очередью получится при использовании нескольких глобальных очередей в одном рантайме, как представленно на изображении далее.

% ТУДУ ПИК

Здесь и далее под рабочей группой будет пониматься глобальная очередь с фиксированным количеством воркеров.

Это было реализовано с следующими измененями в интерефейсе tokio:

\begin{itemize}

\item При создании рантайма пользователь специфицирует количество количество воркеров (\verb|worker_threads|) и количество очередей (\verb|worker_groups|). В результате рантайм инстанциируется с \verb|worker_threads| * \verb|worker_groups| системными потоками для исполнения асинхронных замыканий.
\item При создании задачи c помощью метода \verb|tokio::spawn| выбор рабочей группы осуществляется с помощью локального для потока рандома~\cite{TODO}.
\end{itemize}

Каждая группа изолирована от остальных, то есть воркеры не могут похищать задачи из других глобальных очередей, кроме своей. Это вынуждает их быть запаркованными при нехватке задач в собственной группе и не конфликтовать с другими воркерами.

\subsection{Индентификатор задач}

При создании задач для нее генерируется индентификатор на одной статической ячейке памяти, переменная разделяемая между всеми попытками создать задачу в процессе. То есть при создании каждой задачи, пусть даже в различных потоках, сразуже устанавливается глобальный порядок на одной ячейке памяти для поддержания уникальности индентификаторов.

Этого можно избежать если, создание индентификатора будет состоять из двух шагов.

Первая часть будет отвечать за различие генераторов, второе --- задачи созданные в определенном генераторе.

\subsection{Выбор шарда в OwnedQueue}

При создании задачи ее необходимо зарегистрировать в OwnedQueue, путем выбора шарда и помещению в него указателя Task. Выбор шарда происходит с помощью индентификатора задачи --- путем его преобразвания в \verb|usize|. То есть использование шардов происходило последовательно.

Однако, при наличии составногоо индентификатора повышается возможность конфликта при выборе шарда, так индентификатор теперь менее согласова между различными спавнящими задачи потоками.

Таким образом, следует шарды с использованием заведомо разных индентификаторов.
