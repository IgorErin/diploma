% !TeX spellcheck = ru_RU
% !TEX root = vkr.tex

\section{Проектирование решения}

ТУДУ

\subsection{Использование нескольких рантамов}

После полученных результатов команда \verb|TATLIN.BACKUP| провела эксперименты, где так же стало очевидно преимущество использования нескольких рантаймов.

Однако, неясным осталась надежность такой системы: разработчики tokio не дают никаких гарантий при исполоьзовании нескольких рантаймов, а наоборот утверждают что при некотором обращении может быть достижимо неопределенное поведение.

\verb|TATLIN.BACKUP| позиционирует себя как высоконадежное производительное решение, а потому использовать в нем конструкцию без гарантий никак не возможно.

\subsection{Шардирование глобальной очереди}

Поэтому дупликация ресурсов рантайма была произведена только в области планировщика: глоабльная очередь была дублицирована, как пресдтавленно далее. % TODO

% ТУДУ ПИК

При создании рантайма пользователь специфицирует количество воркеров (\verb|worker_threads|) и количество очередей (\verb|worker_groups|). В результате рантайм инстанциируется с \verb|worker_threads| * \verb|worker_groups| системными потоками для исполнения асинхронных замыканий.

Каждая группа изолирована от остальных, то есть воркеры не могут похищать задачи из других глобальных очередей, кроме своей. И при нехватке задач в собственной группе будут запаркованы.

Создание задачи вне контекста воркера подразумевает помещение ее в одну из глобальных очередей. Выбор очереди происходит с помощью локального для потока рандома уже реализованного в tokio или с специфицированных пользователем значением.

Задачи, созданные воркером попадают в группу воркера.

\subsection{Генерация индентификатора задач}

При создании задач для нее генерируется индентификатор на одной статической ечейке памяти. Этого можно избежать, при использовании состовного индентификатора.

Тогда, создание индентификатора будет состоять из двух шагов --- инстанциации двух частей индетификатора с помощью FAA.

\subsection{Выбор шарда в OwnedQueue}

При создании задачи ее необходимо зарегистрировать в OwnedQueue, путем выбора шарда и помещению в него указателя Task. Выбор шарда происходит с помощью индентификатора задачи --- путем его преобразвания в \verb|usize|. То есть использование шардов происходило последовательно.

Однако, при наличии составногоо индентификатора повышается возможность конфликта при выборе шарда, так индентификатор теперь менее согласова между различными спавнящими задачи потоками.

Таким образом, следует шарды с использованием заведомо разных индентификаторов.
