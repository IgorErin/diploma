% !TeX spellcheck = ru_RU
% !TEX root = vkr.tex

\section{Проектирование решения}

В текущем разделе будут приведены подходы к улучшению многопоточного рантайма.

\subsection{Go lang}

% design doc: http://docs.google.com/document/d/1TTj4T2JO42uD5ID9e89oa0sLKhJYD0Y_kqxDv3I3XMw/edit?tab=t.0

\subsection{Глобальная очередь}

В данном разделе будет более подробно рассмотрена текущая реализация глобальной очереди. Будет приведены требования и свойства к ней. Произведен обзор возможных решений.

\subsection{Текущая реализация}

Глобальная очередь представлена двумя частями:

\begin{itemize}
    \item Synced --- структура требующая синхронизации при доступе. Содержащая два указателя на начало (head) и конец (tail) очереди, флаг обозначающий завершеие работы рантайма (is\_closed)
    \item Shared --- структура не способная использование которой возможно без синхронизации. Она представляет длину очереди (len) и поле пустышка подсказывающее тайпчекеру свойство хранимых в этой коллекции элементов
\end{itemize}

Разделение структуры на две части позволяет агрегировать логически связанные данные вместе при этом использовать для них один мьютекст.

\subsubsection{Выбор алгоритма}

Критирии выбора:

\begin{itemize}
    \item Очередь должна реализовывать порядок FIFO --- для поддержания честного распределения (fairness scheduling)
    \item Произволный размер очереди --- количество задач заранее неизвестно
    \item Переносимость --- не должна зависеть от архитектуры центрального процессора или специализированных ускорителей, например, графических процесоров
\end{itemize}

На ресурсе ACM Digital library по словам \verb|concurrent|, \verb|queue| и \verb|fifo| за последние десять лет, то есть с 2015 по 2025 были выбраны следующие работы:

\begin{itemize}
    \item The State-of-the-Art LCRQ Concurrent Queue Algorithm Does NOT Require CAS2~\cite{LCRQNoCAS2} --- описывает изменение алгоритма~\cite{FastConcurrentQueuesX86} concurrent очереди динамического размера делающее его переносимым (исходная работа предполагала наличие CAS2 инструкций в архитектуре процессора) TOOD(reread)
    \item A lock-free relaxed concurrent queue for fast work distribution~\cite{FastWorkDistribution} --- предлагает очередь ограниченного размера, более подробно описанную в следующей работе TOOD(reread)
    \item A Family of Relaxed Concurrent Queues for Low-Latency Operations and Item Transfers~\cite{FamilyRelaxedConcurrentQueues} --- работа рассматривающая семейство concurrent коллекций ограниченног размер TOOD(reread)
    \item Design and Evaluation of Scalable Concurrent Queues for Many-Core Architectures~\cite{ScalableConcurrentQueuesManyCoreArchitectures}  --- описание реализации очереди ограниченного размера предоставляющей блокирующий и неблокирующий интерфес TOOD(reread)
\end{itemize}

Для реализации был выбран первый~\cite{LCRQNoCAS2} алгоритм, так как он один удоволетворял условию динамического размера очереди.

\subsubsection{Алгоритм LPRQ}

TODO()

\subsubsection{Реализация алгоритма LPRQ}

TODO()

\subsection{Группировака воркеров}
