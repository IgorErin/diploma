% !TeX spellcheck = ru_RU
% !TEX root = vkr.tex

\section{Исследование производительности tokio}

Целями данной главы является изучение изменения производительности при разделениее константного количества потоков производителей между различным количеством инстансов асинхронных рантаймов: если синхронизации исполнителей накладывают ограничение на производительность системы, использующей один асинхронный рантайм, использование нескольких асинхронных рнатаймов с меньшим количество исполнителей и должно продемонстрировать большую пропускную способность.

\subsection{Ход исследования}

Для проверки гипотезы об ограничении производительности ресурсами одного рантайма, были произведены измерения пропускной способности системы, состоящей из нескольких рантаймов, приведенные на графике \ref{fig:tatlin:multi_rt:eval}. Каждое измерение использовало одинаковое количество потоков исполнителей и задач производителей, равномерно разделяемых между рантаймами.

\begin{figure}[H]
    \begin{center}
        \makebox[\textwidth]{\includegraphics[scale=0.90]{pictures/rt_nsapwner128_nspawn1000.png}}
    \end{center}

    \caption{Производительность системы при использовании нескольких рантаймов.}
    \label{fig:tatlin:multi_rt:eval}
\end{figure}

Из изображения заметным становится увеличение производительности при увеличении количества рантаймов, то есть разделение задач между меньшими изолированными группами исполнителей позволяет увеличить пропускную способность.

\subsection{Вывод}

При изоляции меньших групп исполнителей с отдельными глобальными очередями в рамках отдельных инстансов рантаймов наблюдается увеличение пропускной способности. Это может происходить из-за уменьшения количества синхронизаций между исполнителями:

\begin{itemize}
    \item Меньшая борьба за глобальную очередь, за счет деления исполнителей между несколькими очередями.
    \item Меньшее количество исполнителей пытаются украсть задачи у соседей из-за изоляции исполнителей в различных инстансах рантама.
\end{itemize}

Таким образом, шардирование рантаймов позволяет увеличить пропуснкую способность системы.
