% !TeX spellcheck = ru_RU
% !TEX root = vkr.tex

\section{Постановка задачи}
\label{sec:task}

Целью данной работы является проверка гипотезы о бутылочном горлышке, коем явилась по мнению команды \verb|Tatlin.Backup| общая очередь асинхронных событий, и поиску решения этой проблемы. Для чего были поставлены следующие задачи:

\begin{enumerate}
    \item Произвести обзор и выроботку метрик. Рантайм tokio собирает множество метрик в процессе работы --- необходимо выделить значимые.
    \item Создать воспроизводимую систему бенчмарков. Вероятно, совершенно искуственную, однако, позволяющую наблюдать сценарии взаимодейсвия с глобальной очередью.
    \item Произвести анализ собранных метрик. Удостовериться, что глобальная очередь действительно накладывает ограничение на производительность.
    \item Спроектировать решение. Алгоритм шедулинга, в том числе общая очередь, были созданы с оглядкой на реализацию в языке \verb|Go|. Необходимо проверить, что стало достигнуто там за это время.
    \item Прототипровать и анализировать полученные решения.
    \item Интегрировать решение.
\end{enumerate}
