% !TeX spellcheck = ru_RU
% !TEX root = vkr.tex

\section{Постановка задачи}
\label{sec:task}

Целью данной работы является увеличение пропускной способности tokio в типичных для \verb|TATLIN.BACKUP| сценариях. Для чего были поставлены следующие задачи:

\begin{enumerate}
    \item Пронализировать архитектуру tokio для выявления потенциальных критических областей. (сказать что могло быть не так в обзоре)
    \item Разработать систему бенчмарков специфичную для \verb|TATLIN.BACKUP| позволяющую получать воспроизводимый результат.
    \item Произвести анализ собранных метрик и локализовать проблему.
    \item Спроектировать решение основанное на шардировании.
    \item Разработать прототип для анализа применимости подхода.
    \item \setcolor{red}{TODO}: Апробация решения.
\end{enumerate}
