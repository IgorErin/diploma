\section{Асинхронный интерфейс языка Rust}

На листинге~\ref{listing:async_closure} представлена синтаксическая конструкция описывающая асинхронное замыкания в языке \verb|Rust|.

\begin{listing}[H]
    \begin{minted}{rust}
let closure = async { }
    \end{minted}

    \caption{Асинхронное замыкание.}
    \label{listing:async_closure}
\end{listing}

Асинхронное замыкание --- это значение, предстовляющее конечный автомат, сгенирированное компилятором на основании пользовательского кода, использванного в фигурных скобках после ключевого слова \verb|async|. На листинге это значение связано с именем \verb|closure| и представляет собой тривиальный конечный автомат. Каждое сгенерированный компилятором асинхронное замыкание автоматически реализует интерфейс, представленный на листинге~\ref{listing:future_trait}.

\begin{listing}[H]
    \begin{minted}{rust}
trait Future {
    type Output;
    fn poll(self: Pin<&mut Self>, cx: &mut Context)
        -> Poll<Self::Output>;
}
    \end{minted}

    \caption{Интерфейс асинхронных замыканий в языке Rust.}
    \label{listing:future_trait}
\end{listing}

Где метод \verb|poll()| выражает попытку совершить переход от состояния к состоянию и сигнализиует о завершении выполнения замыкания с помощью возвращаемого значения типа представленного на листинге~\ref{listing:future:poll}.

\begin{listing}[H]
    \begin{minted}{rust}
enum Poll<T> { Ready(T), Pending }
    \end{minted}

    \caption{Асинхронное замыкание.}
    \label{listing:future:poll}
\end{listing}

Способного представить результирующее значение асинхронного вычисления в варианте \verb|Ready| или сигнализировать об отсутствии готового результата с вариантом \verb|Pending|.

Метод \verb|poll()| в качестве первого аргумента принимает само асинхронное замыкание, окруженное структурой \verb|Pin|, --- это необходимо для статической гарантии безопасности. В качестве второго аргумента --- контекст, служащий оберткой для значния \verb|Waker|.

\subsection{Waker}

Типичным сценарием использования значения \verb|Waker| является:

\begin{itemize}
    \item Создание с помощью специфичной для рантайма таблицы виртуальных методов.
    \item Передача в метод \verb|poll()| внутри структуры \verb|Context|.
    \item Если метод \verb|poll()| вернул \verb|Pending| --- замыкание должно зависящим от реализации способом сохранить \verb|Waker| для вызова его метода \verb|wake()| при готовности совершать прогресс.
\end{itemize}

То есть \verb|Waker| является связующим звеном между асинхронным ресурсом и исполнителем асинхронных замыканий, сообщающим первому о готовности второго.

\begin{itemize}
    \item Асинхронный ресурс --- листовое асинхронное замыкание, реализованное библиотекой tokio, абстрагирующее архитектурно зависимые интерфейсы, например: \verb|epoll|~\cite{epollLib}, \verb|kqueue|~\cite{kqueue}.
    \item Исполнитель асинхронных замыканий --- в случае tokio планировщик.
\end{itemize}

\subsection{Ключевое слово await}

Ключевое слово \verb|await| позволяет пользователю описать состояние конечного автомата, путем применения его к асинхронному замыканию, как представленно на листинге~\ref{listing:async_sleep}.

\begin{listing}[H]
    \begin{minted}{rust}
async { sleep(10).await }
    \end{minted}

    \caption{Использование ключевого слова await.}
    \label{listing:async_sleep}
\end{listing}

Применение ключевого слова \verb|await| к вызову функции \verb|sleep()|, которая является листовым асинхронным замыканием и предстовляет архитектурно зависимый асинхронный ресурс, генерирует состояние конечного автомата. Исполнение этого замыкания предполагает следующие шаги:

\begin{enumerate}
    \item Первый вызвов метода \verb|poll()| совершит переход от начального состояния к состоянию вызова функции \verb|sleep()|, зарегистрирует структуру \verb|Waker| переданную в вызов метода \verb|poll()| с помощью архитектурно зависимого интерефейса для ожидания истечения десяти секунд.
    \item По истечению десяти секунд, архитектурно зависимый интерфейс сигнализирует о готовности замыкания совершать прогресс путем вызова метода \verb|wake()| на структуре типа \verb|Waker|.
    \item Получив уведомление о готовности асинхронного замыкания, исполнитель продолжит его исполнение с помощью очередного вызова метода \verb|poll()|.
\end{enumerate}

\subsection{Вывод}

Таким образом асинхронные замыкания описанные пользователем с помощью синтаксической конструкции \verb|async|, ключевого слова \verb|await| и библиотечных методов абстрагирующих машинно зависимые детали реализации: таймеры, файловые и сетевые интерефейсы --- подлежат исполнению путем поочередного вызова метода \verb|poll()| после соотвествующего вызова метода \verb|wake()|.
